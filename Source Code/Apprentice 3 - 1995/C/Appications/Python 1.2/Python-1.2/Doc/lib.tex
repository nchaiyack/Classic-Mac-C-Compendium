\documentstyle[twoside,11pt,myformat]{report}

% NOTE: this file controls which chapters/sections of the library
% manual are actually printed.  It is easy to customize your manual
% by commenting out sections that you're not interested in.

\title{Python Library Reference}

TEXTR*ch����

\makeindex			% tell \index to actually write the .idx file


\begin{document}

\pagenumbering{roman}

\maketitle

Copyright \copyright{} 1991-1995 by Stichting Mathematisch Centrum,
Amsterdam, The Netherlands.

\begin{center}
All Rights Reserved
\end{center}

Permission to use, copy, modify, and distribute this software and its
documentation for any purpose and without fee is hereby granted,
provided that the above copyright notice appear in all copies and that
both that copyright notice and this permission notice appear in
supporting documentation, and that the names of Stichting Mathematisch
Centrum or CWI not be used in advertising or publicity pertaining to
distribution of the software without specific, written prior permission.

STICHTING MATHEMATISCH CENTRUM DISCLAIMS ALL WARRANTIES WITH REGARD TO
THIS SOFTWARE, INCLUDING ALL IMPLIED WARRANTIES OF MERCHANTABILITY AND
FITNESS, IN NO EVENT SHALL STICHTING MATHEMATISCH CENTRUM BE LIABLE
FOR ANY SPECIAL, INDIRECT OR CONSEQUENTIAL DAMAGES OR ANY DAMAGES
WHATSOEVER RESULTING FROM LOSS OF USE, DATA OR PROFITS, WHETHER IN AN
ACTION OF CONTRACT, NEGLIGENCE OR OTHER TORTIOUS ACTION, ARISING OUT
OF OR IN CONNECTION WITH THE USE OR PERFORMANCE OF THIS SOFTWARE.

\begin{abstract}

\noindent
Python is an extensible, interpreted, object-oriented programming
language.  It supports a wide range of applications, from simple text
processing scripts to interactive WWW browsers.

While the {\em Python Reference Manual} describes the exact syntax and
semantics of the language, it does not describe the standard library
that is distributed with the language, and which greatly enhances its
immediate usability.  This library contains built-in modules (written
in C) that provide access to system functionality such as file I/O
that would otherwise be inaccessible to Python programmers, as well as
modules written in Python that provide standardized solutions for many
problems that occur in everyday programming.  Some of these modules
are explicitly designed to encourage and enhance the portability of
Python programs.

This library reference manual documents Python's standard library, as
well as many optional library modules (which may or may not be
available, depending on whether the underlying platform supports them
and on the configuration choices made at compile time).  It also
documents the standard types of the language and its built-in
functions and exceptions, many of which are not or incompletely
documented in the Reference Manual.

This manual assumes basic knowledge about the Python language.  For an
informal introduction to Python, see the {\em Python Tutorial}; the
Python Reference Manual remains the highest authority on syntactic and
semantic questions.  Finally, the manual entitled {\em Extending and
Embedding the Python Interpreter} describes how to add new extensions
to Python and how to embed it in other applications.

\end{abstract}

\pagebreak

{
\parskip = 0mm
\tableofcontents
}

\pagebreak

\pagenumbering{arabic}

				% Chapter title:

\input{libintro}		% Introduction

\input{libobjs}			% Built-in Types, Exceptions and Functions
\input{libtypes}
TEXTR*ch����
\input{libfuncs}

\input{libpython}		% Python Services
\input{libsys}
\input{libtypes2}		% types is already taken :-(
\input{libtraceback}
\input{libpickle}
\input{libshelve}
TEXTR*ch����
\input{libmarshal}
\input{libimp}
TEXTR*ch����		% really __builtin__
\input{libmain}			% really __main__

\input{libstrings}		% String Services
\input{libstring}
\input{libregex}
\input{libregsub}
\input{libstruct}

\input{libmisc}			% Miscellaneous Services
\input{libmath}
\input{librand}
\input{libwhrandom}
TEXTR*ch����

TEXTR*ch����		% Generic Operating System Services
\input{libos}
\input{libtime}
\input{libgetopt}
\input{libtempfile}

\input{libsomeos}		% Optional Operating System Services
\input{libsignal}
\input{libsocket}
\input{libselect}
\input{libthread}

\input{libunix}			% UNIX Specific Services
\input{libposix}
\input{libppath}		% == posixpath
\input{libpwd}
\input{libgrp}
TEXTR*ch����
\input{libgdbm}
\input{libtermios}
TEXTR*ch����
\input{libposixfile}

\input{libpdb}			% The Python Debugger

\input{libprofile}		% The Python Profiler

\input{libwww}			% Internet and WWW Services
TEXTR*ch����
\input{liburllib}
\input{libhttplib}
\input{libftplib}
\input{libgopherlib}
\input{libnntplib}
\input{liburlparse}
\input{libhtmllib}
\input{libsgmllib}
\input{librfc822}
\input{libmimetools}

\input{libmm}			% Multimedia Services
TEXTR*ch����
\input{libimageop}
\section{Standard Module \sectcode{aifc}}
\stmodindex{aifc}

This module provides support for reading and writing AIFF and AIFF-C
files.  AIFF is Audio Interchange File Format, a format for storing
digital audio samples in a file.  AIFF-C is a newer version of the
format that includes the ability to compress the audio data.

Audio files have a number of parameters that describe the audio data.
The sampling rate or frame rate is the number of times per second the
sound is sampled.  The number of channels indicate if the audio is
mono, stereo, or quadro.  Each frame consists of one sample per
channel.  The sample size is the size in bytes of each sample.  Thus a
frame consists of \var{nchannels}*\var{samplesize} bytes, and a
second's worth of audio consists of
\var{nchannels}*\var{samplesize}*\var{framerate} bytes.

For example, CD quality audio has a sample size of two bytes (16
bits), uses two channels (stereo) and has a frame rate of 44,100
frames/second.  This gives a frame size of 4 bytes (2*2), and a
second's worth occupies 2*2*44100 bytes, i.e.\ 176,400 bytes.

Module \code{aifc} defines the following function:

\renewcommand{\indexsubitem}{(in module aifc)}
\begin{funcdesc}{open}{file\, mode}
Open an AIFF or AIFF-C file and return an object instance with
methods that are described below.  The argument file is either a
string naming a file or a file object.  The mode is either the string
\code{'r'} when the file must be opened for reading, or \code{'w'}
when the file must be opened for writing.  When used for writing, the
file object should be seekable, unless you know ahead of time how many
samples you are going to write in total and use
\code{writeframesraw()} and \code{setnframes()}.
\end{funcdesc}

Objects returned by \code{aifc.open()} when a file is opened for
reading have the following methods:

\renewcommand{\indexsubitem}{(aifc object method)}
\begin{funcdesc}{getnchannels}{}
Return the number of audio channels (1 for mono, 2 for stereo).
\end{funcdesc}

\begin{funcdesc}{getsampwidth}{}
Return the size in bytes of individual samples.
\end{funcdesc}

\begin{funcdesc}{getframerate}{}
Return the sampling rate (number of audio frames per second).
\end{funcdesc}

\begin{funcdesc}{getnframes}{}
Return the number of audio frames in the file.
\end{funcdesc}

\begin{funcdesc}{getcomptype}{}
Return a four-character string describing the type of compression used
in the audio file.  For AIFF files, the returned value is
\code{'NONE'}.
\end{funcdesc}

\begin{funcdesc}{getcompname}{}
Return a human-readable description of the type of compression used in
the audio file.  For AIFF files, the returned value is \code{'not
compressed'}.
\end{funcdesc}

\begin{funcdesc}{getparams}{}
Return a tuple consisting of all of the above values in the above
order.
\end{funcdesc}

\begin{funcdesc}{getmarkers}{}
Return a list of markers in the audio file.  A marker consists of a
tuple of three elements.  The first is the mark ID (an integer), the
second is the mark position in frames from the beginning of the data
(an integer), the third is the name of the mark (a string).
\end{funcdesc}

\begin{funcdesc}{getmark}{id}
Return the tuple as described in \code{getmarkers} for the mark with
the given id.
\end{funcdesc}

\begin{funcdesc}{readframes}{nframes}
Read and return the next \var{nframes} frames from the audio file.  The
returned data is a string containing for each frame the uncompressed
samples of all channels.
\end{funcdesc}

\begin{funcdesc}{rewind}{}
Rewind the read pointer.  The next \code{readframes} will start from
the beginning.
\end{funcdesc}

\begin{funcdesc}{setpos}{pos}
Seek to the specified frame number.
\end{funcdesc}

\begin{funcdesc}{tell}{}
Return the current frame number.
\end{funcdesc}

\begin{funcdesc}{close}{}
Close the AIFF file.  After calling this method, the object can no
longer be used.
\end{funcdesc}

Objects returned by \code{aifc.open()} when a file is opened for
writing have all the above methods, except for \code{readframes} and
\code{setpos}.  In addition the following methods exist.  The
\code{get} methods can only be called after the corresponding
\code{set} methods have been called.  Before the first
\code{writeframes} or \code{writeframesraw}, all parameters except for
the number of frames must be filled in.

\begin{funcdesc}{aiff}{}
Create an AIFF file.  The default is that an AIFF-C file is created,
unless the name of the file ends in '.aiff' in which case the default
is an AIFF file.
\end{funcdesc}

\begin{funcdesc}{aifc}{}
Create an AIFF-C file.  The default is that an AIFF-C file is created,
unless the name of the file ends in '.aiff' in which case the default
is an AIFF file.
\end{funcdesc}

\begin{funcdesc}{setnchannels}{nchannels}
Specify the number of channels in the audio file.
\end{funcdesc}

\begin{funcdesc}{setsampwidth}{width}
Specify the size in bytes of audio samples.
\end{funcdesc}

\begin{funcdesc}{setframerate}{rate}
Specify the sampling frequency in frames per second.
\end{funcdesc}

\begin{funcdesc}{setnframes}{nframes}
Specify the number of frames that are to be written to the audio file.
If this parameter is not set, or not set correctly, the file needs to
support seeking.
\end{funcdesc}

\begin{funcdesc}{setcomptype}{type\, name}
Specify the compression type.  If not specified, the audio data will
not be compressed.  In AIFF files, compression is not possible.  The
name parameter should be a human-readable description of the
compression type, the type parameter should be a four-character
string.  Currently the following compression types are supported:
NONE, ULAW, ALAW, G722.
\end{funcdesc}

\begin{funcdesc}{setparams}{nchannels\, sampwidth\, framerate\, comptype\, compname}
Set all the above parameters at once.  The argument is a tuple
consisting of the various parameters.  This means that it is possible
to use the result of a \code{getparams} call as argument to
\code{setparams}.
\end{funcdesc}

\begin{funcdesc}{setmark}{id\, pos\, name}
Add a mark with the given id (larger than 0), and the given name at
the given position.  This method can be called at any time before
\code{close}.
\end{funcdesc}

\begin{funcdesc}{tell}{}
Return the current write position in the output file.  Useful in
combination with \code{setmark}.
\end{funcdesc}

\begin{funcdesc}{writeframes}{data}
Write data to the output file.  This method can only be called after
the audio file parameters have been set.
\end{funcdesc}

\begin{funcdesc}{writeframesraw}{data}
Like \code{writeframes}, except that the header of the audio file is
not updated.
\end{funcdesc}

\begin{funcdesc}{close}{}
Close the AIFF file.  The header of the file is updated to reflect the
actual size of the audio data. After calling this method, the object
can no longer be used.
\end{funcdesc}

\input{libjpeg}
\input{librgbimg}

TEXTR*ch����		% Cryptographic Services
\input{libmd5}
\input{libmpz}
\input{librotor}

%TEXTR*ch����		% AMOEBA ONLY

\input{libmac}			% MACINTOSH ONLY
TEXTR*ch����
\input{libmacconsole}
\input{libmacdnr}
\input{libmacfs}
\input{libmactcp}
\input{libmacspeech}

\input{libstdwin}		% STDWIN ONLY

\input{libsgi}			% SGI IRIX ONLY
\section{Built-in Module \sectcode{al}}
\bimodindex{al}

This module provides access to the audio facilities of the SGI Indy
and Indigo workstations.  See section 3A of the IRIX man pages for
details.  You'll need to read those man pages to understand what these
functions do!  Some of the functions are not available in IRIX
releases before 4.0.5.  Again, see the manual to check whether a
specific function is available on your platform.

All functions and methods defined in this module are equivalent to
the C functions with \samp{AL} prefixed to their name.

Symbolic constants from the C header file \file{<audio.h>} are defined
in the standard module \code{AL}, see below.

\strong{Warning:} the current version of the audio library may dump core
when bad argument values are passed rather than returning an error
status.  Unfortunately, since the precise circumstances under which
this may happen are undocumented and hard to check, the Python
interface can provide no protection against this kind of problems.
(One example is specifying an excessive queue size --- there is no
documented upper limit.)

The module defines the following functions:

\renewcommand{\indexsubitem}{(in module al)}

\begin{funcdesc}{openport}{name\, direction\optional{\, config}}
The name and direction arguments are strings.  The optional config
argument is a configuration object as returned by
\code{al.newconfig()}.  The return value is an \dfn{port object};
methods of port objects are described below.
\end{funcdesc}

\begin{funcdesc}{newconfig}{}
The return value is a new \dfn{configuration object}; methods of
configuration objects are described below.
\end{funcdesc}

\begin{funcdesc}{queryparams}{device}
The device argument is an integer.  The return value is a list of
integers containing the data returned by ALqueryparams().
\end{funcdesc}

\begin{funcdesc}{getparams}{device\, list}
The device argument is an integer.  The list argument is a list such
as returned by \code{queryparams}; it is modified in place (!).
\end{funcdesc}

\begin{funcdesc}{setparams}{device\, list}
The device argument is an integer.  The list argument is a list such
as returned by \code{al.queryparams}.
\end{funcdesc}

\subsection{Configuration Objects}

Configuration objects (returned by \code{al.newconfig()} have the
following methods:

\renewcommand{\indexsubitem}{(audio configuration object method)}

\begin{funcdesc}{getqueuesize}{}
Return the queue size.
\end{funcdesc}

\begin{funcdesc}{setqueuesize}{size}
Set the queue size.
\end{funcdesc}

\begin{funcdesc}{getwidth}{}
Get the sample width.
\end{funcdesc}

\begin{funcdesc}{setwidth}{width}
Set the sample width.
\end{funcdesc}

\begin{funcdesc}{getchannels}{}
Get the channel count.
\end{funcdesc}

\begin{funcdesc}{setchannels}{nchannels}
Set the channel count.
\end{funcdesc}

\begin{funcdesc}{getsampfmt}{}
Get the sample format.
\end{funcdesc}

\begin{funcdesc}{setsampfmt}{sampfmt}
Set the sample format.
\end{funcdesc}

\begin{funcdesc}{getfloatmax}{}
Get the maximum value for floating sample formats.
\end{funcdesc}

\begin{funcdesc}{setfloatmax}{floatmax}
Set the maximum value for floating sample formats.
\end{funcdesc}

\subsection{Port Objects}

Port objects (returned by \code{al.openport()} have the following
methods:

\renewcommand{\indexsubitem}{(audio port object method)}

\begin{funcdesc}{closeport}{}
Close the port.
\end{funcdesc}

\begin{funcdesc}{getfd}{}
Return the file descriptor as an int.
\end{funcdesc}

\begin{funcdesc}{getfilled}{}
Return the number of filled samples.
\end{funcdesc}

\begin{funcdesc}{getfillable}{}
Return the number of fillable samples.
\end{funcdesc}

\begin{funcdesc}{readsamps}{nsamples}
Read a number of samples from the queue, blocking if necessary.
Return the data as a string containing the raw data, (e.g., 2 bytes per
sample in big-endian byte order (high byte, low byte) if you have set
the sample width to 2 bytes).
\end{funcdesc}

\begin{funcdesc}{writesamps}{samples}
Write samples into the queue, blocking if necessary.  The samples are
encoded as described for the \code{readsamps} return value.
\end{funcdesc}

\begin{funcdesc}{getfillpoint}{}
Return the `fill point'.
\end{funcdesc}

\begin{funcdesc}{setfillpoint}{fillpoint}
Set the `fill point'.
\end{funcdesc}

\begin{funcdesc}{getconfig}{}
Return a configuration object containing the current configuration of
the port.
\end{funcdesc}

\begin{funcdesc}{setconfig}{config}
Set the configuration from the argument, a configuration object.
\end{funcdesc}

\begin{funcdesc}{getstatus}{list}
Get status information on last error.
\end{funcdesc}

\section{Standard Module \sectcode{AL}}
\nodename{AL (uppercase)}
\stmodindex{AL}

This module defines symbolic constants needed to use the built-in
module \code{al} (see above); they are equivalent to those defined in
the C header file \file{<audio.h>} except that the name prefix
\samp{AL_} is omitted.  Read the module source for a complete list of
the defined names.  Suggested use:

\bcode\begin{verbatim}
import al
from AL import *
\end{verbatim}\ecode

%TEXTR*ch����
TEXTR*ch����
\section{Built-in Module \sectcode{fl}}
\bimodindex{fl}

This module provides an interface to the FORMS Library by Mark
Overmars.  The source for the library can be retrieved by anonymous
ftp from host \samp{ftp.cs.ruu.nl}, directory \file{SGI/FORMS}.  It
was last tested with version 2.0b.

Most functions are literal translations of their C equivalents,
dropping the initial \samp{fl_} from their name.  Constants used by
the library are defined in module \code{FL} described below.

The creation of objects is a little different in Python than in C:
instead of the `current form' maintained by the library to which new
FORMS objects are added, all functions that add a FORMS object to a
form are methods of the Python object representing the form.
Consequently, there are no Python equivalents for the C functions
\code{fl_addto_form} and \code{fl_end_form}, and the equivalent of
\code{fl_bgn_form} is called \code{fl.make_form}.

Watch out for the somewhat confusing terminology: FORMS uses the word
\dfn{object} for the buttons, sliders etc. that you can place in a form.
In Python, `object' means any value.  The Python interface to FORMS
introduces two new Python object types: form objects (representing an
entire form) and FORMS objects (representing one button, slider etc.).
Hopefully this isn't too confusing...

There are no `free objects' in the Python interface to FORMS, nor is
there an easy way to add object classes written in Python.  The FORMS
interface to GL event handling is available, though, so you can mix
FORMS with pure GL windows.

\strong{Please note:} importing \code{fl} implies a call to the GL function
\code{foreground()} and to the FORMS routine \code{fl_init()}.

\subsection{Functions Defined in Module \sectcode{fl}}
\nodename{FL Functions}

Module \code{fl} defines the following functions.  For more information
about what they do, see the description of the equivalent C function
in the FORMS documentation:

\renewcommand{\indexsubitem}{(in module fl)}
\begin{funcdesc}{make_form}{type\, width\, height}
Create a form with given type, width and height.  This returns a
\dfn{form} object, whose methods are described below.
\end{funcdesc}

\begin{funcdesc}{do_forms}{}
The standard FORMS main loop.  Returns a Python object representing
the FORMS object needing interaction, or the special value
\code{FL.EVENT}.
\end{funcdesc}

\begin{funcdesc}{check_forms}{}
Check for FORMS events.  Returns what \code{do_forms} above returns,
or \code{None} if there is no event that immediately needs
interaction.
\end{funcdesc}

\begin{funcdesc}{set_event_call_back}{function}
Set the event callback function.
\end{funcdesc}

\begin{funcdesc}{set_graphics_mode}{rgbmode\, doublebuffering}
Set the graphics modes.
\end{funcdesc}

\begin{funcdesc}{get_rgbmode}{}
Return the current rgb mode.  This is the value of the C global
variable \code{fl_rgbmode}.
\end{funcdesc}

\begin{funcdesc}{show_message}{str1\, str2\, str3}
Show a dialog box with a three-line message and an OK button.
\end{funcdesc}

\begin{funcdesc}{show_question}{str1\, str2\, str3}
Show a dialog box with a three-line message and YES and NO buttons.
It returns \code{1} if the user pressed YES, \code{0} if NO.
\end{funcdesc}

\begin{funcdesc}{show_choice}{str1\, str2\, str3\, but1\optional{\, but2\,
but3}}
Show a dialog box with a three-line message and up to three buttons.
It returns the number of the button clicked by the user
(\code{1}, \code{2} or \code{3}).
\end{funcdesc}

\begin{funcdesc}{show_input}{prompt\, default}
Show a dialog box with a one-line prompt message and text field in
which the user can enter a string.  The second argument is the default
input string.  It returns the string value as edited by the user.
\end{funcdesc}

\begin{funcdesc}{show_file_selector}{message\, directory\, pattern\, default}
Show a dialog box in which the user can select a file.  It returns
the absolute filename selected by the user, or \code{None} if the user
presses Cancel.
\end{funcdesc}

\begin{funcdesc}{get_directory}{}
\funcline{get_pattern}{}
\funcline{get_filename}{}
These functions return the directory, pattern and filename (the tail
part only) selected by the user in the last \code{show_file_selector}
call.
\end{funcdesc}

\begin{funcdesc}{qdevice}{dev}
\funcline{unqdevice}{dev}
\funcline{isqueued}{dev}
\funcline{qtest}{}
\funcline{qread}{}
%\funcline{blkqread}{?}
\funcline{qreset}{}
\funcline{qenter}{dev\, val}
\funcline{get_mouse}{}
\funcline{tie}{button\, valuator1\, valuator2}
These functions are the FORMS interfaces to the corresponding GL
functions.  Use these if you want to handle some GL events yourself
when using \code{fl.do_events}.  When a GL event is detected that
FORMS cannot handle, \code{fl.do_forms()} returns the special value
\code{FL.EVENT} and you should call \code{fl.qread()} to read the
event from the queue.  Don't use the equivalent GL functions!
\end{funcdesc}

\begin{funcdesc}{color}{}
\funcline{mapcolor}{}
\funcline{getmcolor}{}
See the description in the FORMS documentation of \code{fl_color},
\code{fl_mapcolor} and \code{fl_getmcolor}.
\end{funcdesc}

\subsection{Form Objects}

Form objects (returned by \code{fl.make_form()} above) have the
following methods.  Each method corresponds to a C function whose name
is prefixed with \samp{fl_}; and whose first argument is a form
pointer; please refer to the official FORMS documentation for
descriptions.

All the \samp{add_{\rm \ldots}} functions return a Python object representing
the FORMS object.  Methods of FORMS objects are described below.  Most
kinds of FORMS object also have some methods specific to that kind;
these methods are listed here.

\begin{flushleft}
\renewcommand{\indexsubitem}{(form object method)}
\begin{funcdesc}{show_form}{placement\, bordertype\, name}
  Show the form.
\end{funcdesc}

\begin{funcdesc}{hide_form}{}
  Hide the form.
\end{funcdesc}

\begin{funcdesc}{redraw_form}{}
  Redraw the form.
\end{funcdesc}

\begin{funcdesc}{set_form_position}{x\, y}
Set the form's position.
\end{funcdesc}

\begin{funcdesc}{freeze_form}{}
Freeze the form.
\end{funcdesc}

\begin{funcdesc}{unfreeze_form}{}
  Unfreeze the form.
\end{funcdesc}

\begin{funcdesc}{activate_form}{}
  Activate the form.
\end{funcdesc}

\begin{funcdesc}{deactivate_form}{}
  Deactivate the form.
\end{funcdesc}

\begin{funcdesc}{bgn_group}{}
  Begin a new group of objects; return a group object.
\end{funcdesc}

\begin{funcdesc}{end_group}{}
  End the current group of objects.
\end{funcdesc}

\begin{funcdesc}{find_first}{}
  Find the first object in the form.
\end{funcdesc}

\begin{funcdesc}{find_last}{}
  Find the last object in the form.
\end{funcdesc}

%---

\begin{funcdesc}{add_box}{type\, x\, y\, w\, h\, name}
Add a box object to the form.
No extra methods.
\end{funcdesc}

\begin{funcdesc}{add_text}{type\, x\, y\, w\, h\, name}
Add a text object to the form.
No extra methods.
\end{funcdesc}

%\begin{funcdesc}{add_bitmap}{type\, x\, y\, w\, h\, name}
%Add a bitmap object to the form.
%\end{funcdesc}

\begin{funcdesc}{add_clock}{type\, x\, y\, w\, h\, name}
Add a clock object to the form. \\
Method:
\code{get_clock}.
\end{funcdesc}

%---

\begin{funcdesc}{add_button}{type\, x\, y\, w\, h\,  name}
Add a button object to the form. \\
Methods:
\code{get_button},
\code{set_button}.
\end{funcdesc}

\begin{funcdesc}{add_lightbutton}{type\, x\, y\, w\, h\, name}
Add a lightbutton object to the form. \\
Methods:
\code{get_button},
\code{set_button}.
\end{funcdesc}

\begin{funcdesc}{add_roundbutton}{type\, x\, y\, w\, h\, name}
Add a roundbutton object to the form. \\
Methods:
\code{get_button},
\code{set_button}.
\end{funcdesc}

%---

\begin{funcdesc}{add_slider}{type\, x\, y\, w\, h\, name}
Add a slider object to the form. \\
Methods:
\code{set_slider_value},
\code{get_slider_value},
\code{set_slider_bounds},
\code{get_slider_bounds},
\code{set_slider_return},
\code{set_slider_size},
\code{set_slider_precision},
\code{set_slider_step}.
\end{funcdesc}

\begin{funcdesc}{add_valslider}{type\, x\, y\, w\, h\, name}
Add a valslider object to the form. \\
Methods:
\code{set_slider_value},
\code{get_slider_value},
\code{set_slider_bounds},
\code{get_slider_bounds},
\code{set_slider_return},
\code{set_slider_size},
\code{set_slider_precision},
\code{set_slider_step}.
\end{funcdesc}

\begin{funcdesc}{add_dial}{type\, x\, y\, w\, h\, name}
Add a dial object to the form. \\
Methods:
\code{set_dial_value},
\code{get_dial_value},
\code{set_dial_bounds},
\code{get_dial_bounds}.
\end{funcdesc}

\begin{funcdesc}{add_positioner}{type\, x\, y\, w\, h\, name}
Add a positioner object to the form. \\
Methods:
\code{set_positioner_xvalue},
\code{set_positioner_yvalue},
\code{set_positioner_xbounds},
\code{set_positioner_ybounds},
\code{get_positioner_xvalue},
\code{get_positioner_yvalue},
\code{get_positioner_xbounds},
\code{get_positioner_ybounds}.
\end{funcdesc}

\begin{funcdesc}{add_counter}{type\, x\, y\, w\, h\, name}
Add a counter object to the form. \\
Methods:
\code{set_counter_value},
\code{get_counter_value},
\code{set_counter_bounds},
\code{set_counter_step},
\code{set_counter_precision},
\code{set_counter_return}.
\end{funcdesc}

%---

\begin{funcdesc}{add_input}{type\, x\, y\, w\, h\, name}
Add a input object to the form. \\
Methods:
\code{set_input},
\code{get_input},
\code{set_input_color},
\code{set_input_return}.
\end{funcdesc}

%---

\begin{funcdesc}{add_menu}{type\, x\, y\, w\, h\, name}
Add a menu object to the form. \\
Methods:
\code{set_menu},
\code{get_menu},
\code{addto_menu}.
\end{funcdesc}

\begin{funcdesc}{add_choice}{type\, x\, y\, w\, h\, name}
Add a choice object to the form. \\
Methods:
\code{set_choice},
\code{get_choice},
\code{clear_choice},
\code{addto_choice},
\code{replace_choice},
\code{delete_choice},
\code{get_choice_text},
\code{set_choice_fontsize},
\code{set_choice_fontstyle}.
\end{funcdesc}

\begin{funcdesc}{add_browser}{type\, x\, y\, w\, h\, name}
Add a browser object to the form. \\
Methods:
\code{set_browser_topline},
\code{clear_browser},
\code{add_browser_line},
\code{addto_browser},
\code{insert_browser_line},
\code{delete_browser_line},
\code{replace_browser_line},
\code{get_browser_line},
\code{load_browser},
\code{get_browser_maxline},
\code{select_browser_line},
\code{deselect_browser_line},
\code{deselect_browser},
\code{isselected_browser_line},
\code{get_browser},
\code{set_browser_fontsize},
\code{set_browser_fontstyle},
\code{set_browser_specialkey}.
\end{funcdesc}

%---

\begin{funcdesc}{add_timer}{type\, x\, y\, w\, h\, name}
Add a timer object to the form. \\
Methods:
\code{set_timer},
\code{get_timer}.
\end{funcdesc}
\end{flushleft}

Form objects have the following data attributes; see the FORMS
documentation:

\begin{tableiii}{|l|c|l|}{code}{Name}{Type}{Meaning}
  \lineiii{window}{int (read-only)}{GL window id}
  \lineiii{w}{float}{form width}
  \lineiii{h}{float}{form height}
  \lineiii{x}{float}{form x origin}
  \lineiii{y}{float}{form y origin}
  \lineiii{deactivated}{int}{nonzero if form is deactivated}
  \lineiii{visible}{int}{nonzero if form is visible}
  \lineiii{frozen}{int}{nonzero if form is frozen}
  \lineiii{doublebuf}{int}{nonzero if double buffering on}
\end{tableiii}

\subsection{FORMS Objects}

Besides methods specific to particular kinds of FORMS objects, all
FORMS objects also have the following methods:

\renewcommand{\indexsubitem}{(FORMS object method)}
\begin{funcdesc}{set_call_back}{function\, argument}
Set the object's callback function and argument.  When the object
needs interaction, the callback function will be called with two
arguments: the object, and the callback argument.  (FORMS objects
without a callback function are returned by \code{fl.do_forms()} or
\code{fl.check_forms()} when they need interaction.)  Call this method
without arguments to remove the callback function.
\end{funcdesc}

\begin{funcdesc}{delete_object}{}
  Delete the object.
\end{funcdesc}

\begin{funcdesc}{show_object}{}
  Show the object.
\end{funcdesc}

\begin{funcdesc}{hide_object}{}
  Hide the object.
\end{funcdesc}

\begin{funcdesc}{redraw_object}{}
  Redraw the object.
\end{funcdesc}

\begin{funcdesc}{freeze_object}{}
  Freeze the object.
\end{funcdesc}

\begin{funcdesc}{unfreeze_object}{}
  Unfreeze the object.
\end{funcdesc}

%\begin{funcdesc}{handle_object}{} XXX
%\end{funcdesc}

%\begin{funcdesc}{handle_object_direct}{} XXX
%\end{funcdesc}

FORMS objects have these data attributes; see the FORMS documentation:

\begin{tableiii}{|l|c|l|}{code}{Name}{Type}{Meaning}
  \lineiii{objclass}{int (read-only)}{object class}
  \lineiii{type}{int (read-only)}{object type}
  \lineiii{boxtype}{int}{box type}
  \lineiii{x}{float}{x origin}
  \lineiii{y}{float}{y origin}
  \lineiii{w}{float}{width}
  \lineiii{h}{float}{height}
  \lineiii{col1}{int}{primary color}
  \lineiii{col2}{int}{secondary color}
  \lineiii{align}{int}{alignment}
  \lineiii{lcol}{int}{label color}
  \lineiii{lsize}{float}{label font size}
  \lineiii{label}{string}{label string}
  \lineiii{lstyle}{int}{label style}
  \lineiii{pushed}{int (read-only)}{(see FORMS docs)}
  \lineiii{focus}{int (read-only)}{(see FORMS docs)}
  \lineiii{belowmouse}{int (read-only)}{(see FORMS docs)}
  \lineiii{frozen}{int (read-only)}{(see FORMS docs)}
  \lineiii{active}{int (read-only)}{(see FORMS docs)}
  \lineiii{input}{int (read-only)}{(see FORMS docs)}
  \lineiii{visible}{int (read-only)}{(see FORMS docs)}
  \lineiii{radio}{int (read-only)}{(see FORMS docs)}
  \lineiii{automatic}{int (read-only)}{(see FORMS docs)}
\end{tableiii}

\section{Standard Module \sectcode{FL}}
\nodename{FL (uppercase)}
\stmodindex{FL}

This module defines symbolic constants needed to use the built-in
module \code{fl} (see above); they are equivalent to those defined in
the C header file \file{<forms.h>} except that the name prefix
\samp{FL_} is omitted.  Read the module source for a complete list of
the defined names.  Suggested use:

\bcode\begin{verbatim}
import fl
from FL import *
\end{verbatim}\ecode

\section{Standard Module \sectcode{flp}}
\stmodindex{flp}

This module defines functions that can read form definitions created
by the `form designer' (\code{fdesign}) program that comes with the
FORMS library (see module \code{fl} above).

For now, see the file \file{flp.doc} in the Python library source
directory for a description.

XXX A complete description should be inserted here!
TEXTR*ch����
\input{libgl}
\input{libimgfile}
%\input{libpanel}

\input{libsun}			% SUNOS ONLY

TEXTR*ch����			% Index

\end{document}
