\section{Built-in Module \sectcode{fm}}
\bimodindex{fm}

This module provides access to the IRIS {\em Font Manager} library.
It is available only on Silicon Graphics machines.
See also: 4Sight User's Guide, Section 1, Chapter 5: Using the IRIS
Font Manager.

This is not yet a full interface to the IRIS Font Manager.
Among the unsupported features are: matrix operations; cache
operations; character operations (use string operations instead); some
details of font info; individual glyph metrics; and printer matching.

It supports the following operations:

\renewcommand{\indexsubitem}{(in module fm)}
\begin{funcdesc}{init}{}
Initialization function.
Calls \code{fminit()}.
It is normally not necessary to call this function, since it is called
automatically the first time the \code{fm} module is imported.
\end{funcdesc}

\begin{funcdesc}{findfont}{fontname}
Return a font handle object.
Calls \code{fmfindfont(\var{fontname})}.
\end{funcdesc}

\begin{funcdesc}{enumerate}{}
Returns a list of available font names.
This is an interface to \code{fmenumerate()}.
\end{funcdesc}

\begin{funcdesc}{prstr}{string}
Render a string using the current font (see the \code{setfont()} font
handle method below).
Calls \code{fmprstr(\var{string})}.
\end{funcdesc}

\begin{funcdesc}{setpath}{string}
Sets the font search path.
Calls \code{fmsetpath(string)}.
(XXX Does not work!?!)
\end{funcdesc}

\begin{funcdesc}{fontpath}{}
Returns the current font search path.
\end{funcdesc}

Font handle objects support the following operations:

\renewcommand{\indexsubitem}{(font handle method)}
\begin{funcdesc}{scalefont}{factor}
Returns a handle for a scaled version of this font.
Calls \code{fmscalefont(\var{fh}, \var{factor})}.
\end{funcdesc}

\begin{funcdesc}{setfont}{}
Makes this font the current font.
Note: the effect is undone silently when the font handle object is
deleted.
Calls \code{fmsetfont(\var{fh})}.
\end{funcdesc}

\begin{funcdesc}{getfontname}{}
Returns this font's name.
Calls \code{fmgetfontname(\var{fh})}.
\end{funcdesc}

\begin{funcdesc}{getcomment}{}
Returns the comment string associated with this font.
Raises an exception if there is none.
Calls \code{fmgetcomment(\var{fh})}.
\end{funcdesc}

\begin{funcdesc}{getfontinfo}{}
Returns a tuple giving some pertinent data about this font.
This is an interface to \code{fmgetfontinfo()}.
The returned tuple contains the following numbers:
{\tt(\var{printermatched}, \var{fixed_width}, \var{xorig}, \var{yorig},
\var{xsize}, \var{ysize}, \var{height}, \var{nglyphs})}.
\end{funcdesc}

\begin{funcdesc}{getstrwidth}{string}
Returns the width, in pixels, of the string when drawn in this font.
Calls \code{fmgetstrwidth(\var{fh}, \var{string})}.
\end{funcdesc}
ating a file \samp{spammodule.c}.  (In general, if a
module is called \samp{spam}, the C file containing its implementation
is called \file{spammodule.c}; if the module name is very long, like
\samp{spammify}, the module name can be just \file{spammify.c}.)

The first line of our file can be:

\begin{verbatim}
    #include "Python.h"
\end{verbatim}

which pulls in the Python API (you can add a comment describing the
purpose of the module and a copyright notice if you like).

All user-visible symbols defined by \code{"Python.h"} have a prefix of
\samp{Py} or \samp{PY}, except those defined in standard header files.
For convenience, and since they are used extensively by the Python
interpreter, \code{"Python.h"} includes a few standard header files:
\code{<stdio.h>}, \code{<string.h>}, \code{<errno.h>}, and
\code{<stdlib.h>}.  If the latter header file does not exist on your
system, it declares the functions \code{malloc()}, \code{free()} and
\code{realloc()} directly.

The next thing we add to our module file is the C function that will
be called when the Python expression \samp{spam.system(\var{string})}
is evaluated (we'll see shortly how it ends up being called):

\begin{verbatim}
    static PyObject *
    spam_system(self, args)
        PyObject *self;
        PyObject *args;
    {
        char *command;
        int sts;
        if (!PyAg\chapter{Cryptographic Services}
\index{cryptography}

The modules described in this chapter implement various algorithms of
a cryptographic nature.  They are available at the discretion of the
installation.  Here's an overview:

\begin{description}

\item[md5]
--- RSA's MD5 message digest algorithm.

\item[mpz]
--- Interface to the GNU MP library for arbitrary precision arithmetic.

\item[rotor]
--- Enigma-like encryption and decryption.

\end{description}

Hardcore cypherpunks will probably find the Python Cryptography Kit of
further interest; the package adds built-in modules for DES and IDEA
encryption, and provides a Python module for reading and decrypting
PGP files.  The Python Cryptography Kit is not distributed with Python
but available separately.  See the URL
\file{http://www.cs.mcgill.ca/\%7Efnord/crypt.html} for more information.
\index{PGP}
\indexii{DES}{cipher}
\indexii{IDEA}{cipher}
\index{Python Cryptography Kit}
-- in order to do anything with them in our C function we have
to convert them to C values.  The function \code{PyArg_ParseTuple()}
in the Python API checks the argument types and converts them to C
values.  It uses a template string to determine the required types of
the arguments as well as the types of the C variables into which to
store the converted values.  More about this later.

\code{PyArg_ParseTuple()} returns true (nonzero) if all arguments have
the right type and its components have been stored in the variables
whose addresses are passed.  It returns false (zero) if an invalid
argument list was passed.  In the latter case it also raises an
appropriate exception by so the calling function can return
\code{NULL} immediately (as we saw in the example).


\section{Intermezzo: Errors and Exceptions}

An important convention throughout the Python interpreter is the
following: when a function fails, it should set an exception condition
and return an error value (usually a \code{NULL} pointer).  Exceptions
are stored in a static global variable inside the interpreter; if this
variable is \code{NULL} no exception has occurred.  A second global
variable stores the ``associated value'' of the exception (the second
argument to \code{raise}).  A third variable contains the stack
traceback in case the error originated in Python code.  These three
variables are the C equivalents of the Python variables
\code{sys.exc_type}, \code{sys.exc_value} and \code{sys.exc_traceback}
(see the section on module \code{sys} in the Library Reference
Manual).  It is important to know about them to understand how errors
are passed around.

The Python API defines a number of functions to set various types of
exceptions.

The most common one is \code{PyErr_SetString()}.  Its arguments are an
exception object and a C string.  The exception object is usually a
predefined object like \code{PyExc_ZeroDivisionError}.  The C string
indicates the cause of the error and is converted to a Python string
object and stored as the ``associated value'' of the exception.

Another useful function is \code{PyErr_SetFromErrno()}, which only
takes an exception argument and constructs the associated value by
inspection of the (\UNIX{}) global variable \code{errno}.  The most
general function is \code{PyErr_SetObject()}, which takes two object
arguments, the exception and its associated value.  You don't need to
\code{Py_INCREF()} the objects passed to any of these functions.

You can test non-destructively whether an exception has been set with
\code{PyErr_Occurred()}.  This returns the current exception object,
or \code{NULL} if no exception has occurred.  You normally don't need
to call \code{PyErr_Occurred()} to see whether an error occurred in a
function call, since you should be able to tell from the return value.

When a function \var{f} that calls another function var{g} detects
that the latter fails, \var{f} should itself return an error value
(e.g. \code{NULL} or \code{-1}).  It should \emph{not} call one of the
\code{PyErr_*()} functions --- one has already been called by \var{g}.
\var{f}'s caller is then supposedt\section{Built-in Module \sectcode{audio}}
\bimodindex{audio}

\strong{Note:} This module is obsolete, since the hardware to which it
interfaces is obsolete.  For audio on the Indigo or 4D/35, see
built-in module \code{al} above.

This module provides rudimentary access to the audio I/O device
\file{/dev/audio} on the Silicon Graphics Personal IRIS 4D/25;
see {\it audio}(7). It supports the following operations:

\renewcommand{\indexsubitem}{(in module audio)}
\begin{funcdesc}{setoutgain}{n}
Sets the output gain.
\iftexi
\code{0 <= \var{n} < 256}.
\else
$0 \leq \var{n} < 256$.
%%JHXXX Sets the output gain (0-255).
\fi
\end{funcdesc}

\begin{funcdesc}{getoutgain}{}
Returns the output gain.
\end{funcdesc}

\begin{funcdesc}{setrate}{n}
Sets the sampling rate: \code{1} = 32K/sec, \code{2} = 16K/sec,
\code{3} = 8K/sec.
\end{funcdesc}

\begin{funcdesc}{setduration}{n}
Sets the `sound duration' in units of 1/100 seconds.
\end{funcdesc}

\begin{funcdesc}{read}{n}
Reads a chunk of
\var{n}
sampled bytes from the audio input (line in or microphone).
The chunk is returned as a string of length n.
Each byte encodes one sample as a signed 8-bit quantity using linear
encoding.
This string can be converted to numbers using \code{chr2num()} described
below.
\end{funcdesc}

\begin{funcdesc}{write}{buf}
Writes a chunk of samples to the audio output (speaker).
\end{funcdesc}

These operations support asynchronous audio I/O:

\renewcommand{\indexsubitem}{(in module audio)}
\begin{funcdesc}{start_recording}{n}
Starts a second thread (a process with shared memory) that begins reading
\var{n}
bytes from the audio device.
The main thread immediately continues.
\end{funcdesc}

\begin{funcdesc}{wait_recording}{}
Waits for the second thread to finish and returns the data read.
\end{funcdesc}

\begin{funcdesc}{stop_recording}{}
Makes the second thread stop reading as soon as possible.
Returns the data read so far.
\end{funcdesc}

\begin{funcdesc}{poll_recording}{}
Returns true if the second thread has finished reading (so
\code{wait_recording()} would return the data without delay).
\end{funcdesc}

\begin{funcdesc}{start_playing}{}
\funcline{wait_playing}{}
\funcline{stop_playing}{}
\funcline{poll_playing}{}
\begin{sloppypar}
Similar but for output.
\code{stop_playing()}
returns a lower bound for the number of bytes actually played (not very
accurate).
\end{sloppypar}
\end{funcdesc}

The following operations do not affect the audio device but are
implemented in C for efficiency:

\renewcommand{\indexsubitem}{(in module audio)}
\begin{funcdesc}{amplify}{buf\, f1\, f2}
Amplifies a chunk of samples by a variable factor changing from
\code{\var{f1}/256} to \code{\var{f2}/256.}
Negative factors are allowed.
Resulting values that are to large to fit in a byte are clipped.         
\end{funcdesc}

\begin{funcdesc}{reverse}{buf}
Returns a chunk of samples backwards.
\end{funcdesc}

\begin